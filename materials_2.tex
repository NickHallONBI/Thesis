\chapter{Enabling Structured Illumination Microscopy at greater depth}

\section{Sample Induced Aberrations in Thick Samples}
\label{sec:sample_aberrations_thick}

The origin of optical aberrations and their effect on image
formation from a geometric optics standpoint and how this leads 
decreased image quality and resolution has already been covered 
in Section~\ref{sec:aberrations}. The extent of optical 
aberrations in thick samples and the particular effects of
aberrations on structured illumination microscopy (SIM) imaging 
are worth paying particular attention to. 

In widefield microscopy, an even field of illumination is 
applied to the field of view and the resultant fluorescent
signal from the sample is measured. In SIM, this even 
illumination is replaced by a sinusoidally varying
illumination pattern. Multiple images with varying angle 
and phase shifts to the illumination pattern are acquired. 
These images are then used to reconstruct a single 
super-resolved image.\cite{gustafsson2000surpassing,gustafsson2008three}
Recall a single fluorescent image, $F(\bar{x})$, is 
defined by:

\begin{equation}\label{eq:SIM_fluorescent_image}
\begin{split}
	F(\bar{x}) &= (E \circledast H)(\bar{x})\\
	F(\bar{x}) &= [D(\bar{x})I(\bar{x})] \circledast H(\bar{x})\\
\end{split}
\end{equation}

Where $D(\bar{x})$ is the sample fluorescence distribution,
$I(\bar{x})$ is the illumination pattern, $E(\bar{x})$ is the 
resultant emission signal and $H(\bar{x})$ is the system point
spread function (PSF). Section~\ref{sec:aberrations} has already
discussed the effect aberrations have on distorting the system
PSF, $H(\bar{x})$. However, in SIM imaging the illumination 
pattern, $I(\bar{x})$, is also formed in the same imaging path
and is therefore the illumination pattern formation is also 
affected by the system aberrations.

Consider the illumination pattern function. Fully coherent
illumination conditions yeild optimum imaging performance and
are therefore chosen. In practice, only partially coherent 
conditions are achieveable. Utilising the fully coherent 
approximation, $I(\bar{x})$ can be expressed as:

\begin{equation}\label{eq:SIM_illumination}
	I(\bar{x}) = \left\| \frac{1}{2}[1 + \cos(\bar{x}\bullet\bar{p} + \phi)]\circledast H_{amp}(\bar{x}) \right\|^{2}
\end{equation}

Where $\bar{p}$ and $\phi$ define the orientation and phase of
the illumination pattern respectively. $H_{amp}(\bar{x})$ is the
amplitude PSF, as opposed to $H(\bar{x})$ which is the intensity
PSF and is defined as:

\begin{equation}\label{eq:amplitude_PSF}
	H_{amp}(\bar{x}) = \mathcal{F}[P(\bar{r})]
\end{equation}

Where $P$ is the pupil function, $\bar{r}$ is the coordinate
vector in the pupil plane and $\mathcal{F}$ is the Fourier
transform operation. Utiliting the Fourier convolution 
theorem, Equation~\ref{eq:SIM_illumination} becomes:

\begin{equation}\label{eq:SIM_illumination_invFT}
I(\bar{x}) = \left\| \mathcal{F}^{-1}\left\{\frac{1}{2}\left[\delta(\bar{r}) + \frac{\exp^{-i\theta}}{2}\delta(\bar{r}-\bar{p}) + \frac{\exp^{i\theta}}{2}\delta(\bar{r}+\bar{p})\right]P(\bar{r})\right\} \right\|^{2}
\end{equation}

Where value of $\bar{p}$ positions the delta functions in the
pupil plane. $\bar{p}$ is chosen to have a value of or close
to 1, to maximise the resolution increase. 
Equation~\ref{eq:SIM_illumination_invFT} there is clearly only
illumination at three points in the pupil plane, at the centre 
and diametrically opposed at the edge of the pupil. There are
two principle consequences of this fact. Firstly, since the 
illumination only exists at the centre and edges of the pupil
plane, the illumination pattern is only affected by phase 
variations at those points. Therefore, the illumination pattern
should be immune to aberrations with little phase variation at 
the edges of the pupil. Secondly, rotationally symmetrical 
aberrations, such as spherical aberration, have have the main
effect of refocusing the illumination pattern.\cite{booth2015aberrations}

It has also been shown that the imaging properties of a SIM
system and the quality of the reconstructed super-resolution
images is determined by the system's ability to reliably
image the high spatial frequencies.\cite{debarre2008adaptive,thomas2015enhanced}
The aberrations which affect the strength of high spatial 
frequency signals can be divided into two categories; 
isoplanar and anisoplanar aberrations. Single adaptive 
elements, such as the deformable mirror in the DeepSIM 
setup, can only correct for the isoplanar aberrations.
Anisoplanar aberrations will still introduce artifacts 
in the final, resonstructed data. However, provided that
the anisoplanar aberrations are considerably smaller than
the isoplanar aberrations, these aberrations will be small
compared to the SIM image and will not prevent a successful
reconstruction.\cite{thomas2015enhanced}

The cummulative effects of the optical aberrations is that
traditional SIM imaging is limited in its imaging depth to
approximately $10\mu m$.\cite{wu2018faster} However, 
correcting the optical aberrations present would not yeild
an infinite depth of imaging due to scattering of photons,
which increases with depth. Nonetheless, by correcting for
optical aberrations present DeepSIM offers the potential to 
acquire super-resolution SIM images at an unprecidented depth.

\section{Sensorless Adaptive Optics}
\label{sec:sensorless_AO}

\subsection{Image Quality Metric: Fourier Power}
\label{subsec:fourier_power_metric}

As mentioned in Section~\ref{subsubsec:sensorless_correction}, 
sensorless AO correction requires an image quality metric which
is well suited to both the imaging modality and the desired 
sample. Microscope-AOtools implements a suite of image quality
metrics to choose from. Since SIM data is contingent on high
quality Fourier information, a sensible starting point for 
the image quality metric, $S$, is the total power of the Fourier
spectrum:

\begin{equation}\label{eq:fourier_power_spectrum}
S = \sum\limits_{n,m}{\left\| \tilde{\textbf{D}}_{n,m} \right\|^2}
\end{equation}

Where $\tilde{\textbf{D}}$ is the 2D discrete Fourier transform of 
the observed fluorescent signal (i.e. the camera image)and $n$ and 
$m$ are the pixel coordinates with ranges $[0, N-1]$ and $[0, M-1]$ 
respectively, where $N$ and $M$ are the number of pixels along each 
dimension of the image. From Parseval's Theorem, $\sum\limits_{n,m}
\left\| \tilde{\textbf{D}}_{n,m} \right\|^2$ is the power of the 
observed signal. However, the Fourier spectrum obtained is not 
entirely composed of sample information. Recall from 
Equation\ref{eq:SIM_fluorescent_image} the fluorescent signal is the 
convolution of the distribution of fluorescent emission and the system 
PSF. For a widefield image, $E(\bar{x}) = D(\bar{x})$. Applying this
substitution and a Fourier transfrom yields:
\begin{equation}\label{eq:fluor_signal_fourier}
\begin{split}
	\mathcal{F}[\textbf{F}(\bar{x})] &= \mathcal{F}[\textbf{D}(\bar{x}) \circledast \textbf{H}(\bar{x})]\\
	\tilde{\textbf{F}}(\bar{k}) &= \tilde{\textbf{D}}(\bar{k}) \tilde{\textbf{H}}(\bar{k})\\
	&= \tilde{\textbf{D}}(\bar{k}) \textbf{O}(\bar{k})		
\end{split}
\end{equation}

Where the tilde ($\sim$) represent the Fourier transform of the corresponding 
real-space functions and $\textbf{O}(\bar{k})$ is the system optical transfer 
function (OTF).\cite{gustafsson2008three} The OTF both attenuates the sample 
fluorescence distribution spatial frequencies and acts as a lowpass filter. 
Since $\tilde{\textbf{D}}_{n,m}$ is the 2D discrete Fourier transform of 
the observed data, this lowpass filter, $\mu_{n,m}$, is defined as:

\begin{equation}\label{eq:circular_mask}
\mu_{n,m} = 
\begin{cases}
1, & \sqrt{n'^{2} + m'^{2}} \le w\\
0, & \sqrt{n'^{2} + m'^{2}} > w\\ 
\end{cases}
\end{equation}

Where $n' = n - \frac{N-1}{2}$, $m' = m - \frac{M-1}{2}$ and $w$ a radius 
defined as the size of the field of view of the image divided by the Abbe
diffraction limit for the system, $\frac{1.22\lambda}{2NA}$.\cite{abbe1873beitrage} 

Within the OTF lowpass limit, certain spatial frequencies will be dominated
by noise contributions. As such, when attempting to quantify the power 
contribution from the fluorescent it is desirable to ignore the noise
contributions. A thresholding mask, $\tau_{n,m}$ is created and defined as:

\begin{equation}\label{eq:noise_threshold_mask}
\tau_{n,m} = 
\begin{cases}
0, & \left\| \tilde{\textbf{D}}_{n,m} \right\|^2 < \delta\\
1, & \left\| \tilde{\textbf{D}}_{n,m} \right\|^2 \ge \delta\\ 
\end{cases}
\end{equation}

Where $\delta$ is a noise threshold defined as:

\begin{equation}\label{eq:noise_threshold}
\delta = \frac{\alpha}{P_{noise}}\sum\limits_{l,k}{\left\| \tilde{\textbf{D}}_{l,k} \right\|^2}
\end{equation}

Where $\left\| \tilde{\textbf{D}}_{l,k} \right\|^2$ is the power in the 
$l$-th, $k$-th spatial frequency where $l$ \& $k$ are the coordinates 
where $\mu_{n,m} = 0$, $P_{noise}$ is the total number of pixels 
outside the OTF lowpass limit and $\alpha$ is a user defined 
amplification factor. In essence, the threshold is the mean of the
power of the spatial frequencies consisting entirely of noise signals,
scaled by $\alpha$.	The value of $\alpha$ must be chosen to threshold
the spatial frequencies dominated by noise without losing the sensitivity
to true, albeit low amplitude, spatial frequencies. Empirically, an 
$\alpha = 1.125$ offers a reasonable compromise between these traits.

Finally, the bulk of the improvement in spatial frequency content will
occur in the mid to high frequencies (relative to $w$). As such there is 
a desire for increased sensitivity to the power fluctuations in these 
spatial frequencies. A modulation mask, $\omega$, is constructed and 
defined as:

\begin{equation}\label{eq:attenuation_mask_norm_and_scaled}
\omega_{n,m} = \beta \left| \left(1 - e^{\frac{\sqrt{n'^{2} + m'^{2}}}{w}-1}\right) \left(n'^{2} + m'^{2}\right) \right|
\end{equation}

Where $w$, $n'$ and $m'$ are defined as before, $\left| \textbf{x} \right|$ 
is $\textbf{x}$ normalised to $[0,1]$, and $\beta$ is a user defined 
normalisation factor. In effect, this modulation mask amplifies the mid 
to high spatial frequencies while attenuating the low spatial frequencies 
and the high spatial frequency noise close to $w$.

Applying these modification to Equation~\ref{eq:fourier_power_spectrum} 
yields the image quality metric:

\begin{equation}\label{eq:Fourier_power_metric}
S = \sum\limits_{n,m}{\mu_{n,m}\tau_{n,m}\omega_{n,m}\left\| \tilde{\textbf{D}}_{n,m} \right\|^2}
\end{equation}

Where $\mu_{n,m}$, $\tau_{n,m}$ and $\omega_{n,m}$ are defined as in
Equations~\ref{eq:circular_mask},~\ref{eq:noise_threshold_mask} and
\ref{eq:attenuation_mask_norm_and_scaled} respectively and 
$\tilde{\textbf{D}}_{n,m}$ is defined as before. In practice, $S$
measures the total power of the sample spatial frequencies, with 
increased sensitivity to mid to high spatial frequency changes. As
such, this image quality metric is called the Fourier Power metric.
As previously mentioned, SIM imaging depends on high quality spatial 
frequency information.\cite{debarre2008adaptive,thomas2015enhanced} 
Since the Fourier Power imaging metric maximises the intensity of 
the mid-high spatial frequency content, this image quality metric 
is particularly well suited to correcting aberrations for SIM data.

\subsection{IsoSense}
\label{subsec:isosense}

Anisotropies in the sample fluorescence distribution can bias 
the corrections towards improving the image quality in a 
non-uniform manner. There has recently been a technique developed 
to overcome  this issue; IsoSense\cite{vzurauskas2019isosense}. 
It relies  on producing spatially structured light in order to fill 
empty sections of the image Fourier spectrum. IsoSense is designed 
to be used in structured illumination microscopy (SIM) setups since 
they often incorporate spatial light modulators (SLM) as high-speed, 
dynamic diffraction  gratings and SIM is particularly sensitive to 
Fourier space anisotropies.

Microscope-AOtools incorporates the methods necessary to implement
IsoSense. Figure~\ref{fig:isosense_visualisation} shows both
the structured illumination pattern applied to the SLM 
and the location of the beams in Fourier space. The illumination
pattern shown in Figure~\ref{fig:isosense_visualisation_real} is
the inverse Fourier transform of the 4-beam interference pattern 
in Figure~\ref{fig:isosense_visualisation_ft}. The location 
of these beams, $\bar{\kappa}$, are: 
$(0,0)$, $(0,\gamma w)$, $(0,-\gamma w)$, $(\gamma w, 0)$, 
$(-\gamma w, 0)$, $(\frac{\gamma w}{2}, \frac{\gamma w}{2})$, 
$(-\frac{\gamma w}{2}, \frac{\gamma w}{2})$, $(\frac{\gamma w}{2},
-\frac{\gamma w}{2})$, $(-\frac{\gamma w}{2}, 
-\frac{\gamma w}{2})$. $u,v$ are the spatial frequency 
analogues of the $x,y$ axes. $w$ is the Abbe diffraction limit and 
$\gamma$ is a user defined fill fraction. This fill
fraction controls the positions of the beams in the interference
pattern and hence the region of the Fourier spectrum which will 
be enhanced over normal illumination. The resultant emission image 
obtained, $E(\bar{x})$, can be described as:

\begin{equation}\label{eq:isosense_real}
E(\bar{x}) = D(\bar{x}) \times I(\bar{x})
\end{equation}	

Where $I(\bar{x})$ is the interference pattern, similar to that shown in 
Figure\ref{fig:isosense_visualisation_real}, $D(\bar{x})$ is the sample 
fluorescence distribution as before and $\bar{x}$ is the spatial 
coordinate vector. Applying a Fourier transform yields:

\begin{equation}\label{eq:isosense_ft}
\begin{split}
\mathcal{F}[F(\bar{x})] &= \mathcal{F}[D(\bar{x})\times I(\bar{x})] \\
\tilde{\textbf{F}}(\bar{k}) &= \tilde{\textbf{D}}(\bar{k}) \circledast \tilde{\textbf{I}}(\bar{k}) \\
\tilde{\textbf{F}}(\bar{k}) &= \sum_{\bar{\kappa}}\iint\tilde{\textbf{D}}(\bar{k})\delta(\bar{k} - \bar{\kappa})dudv \\
\tilde{\textbf{F}}(\bar{k}) &= \sum_{\bar{\kappa}}\tilde{\textbf{D}}(\bar{k} - \bar{\kappa})
\end{split}
\end{equation}

Where $\tilde{\textbf{F}}(\bar{k})$, $\tilde{\textbf{I}}(\bar{k})$ and 
$\tilde{\textbf{D}}(\bar{k})$ are the Fourier transforms of $F(\bar{x})$, 
$I(\bar{x})$ and $D(\bar{x})$, the $\bar{\kappa}$ frequencies are the 
locations of interference pattern beams described above and $\bar{k}$ is 
the spatial frequency vector. By convolving the sample fluorescence 
distribution with  multiple $\delta$-functions, multiple copies of the 
sample spatial frequency information are created and regions of the 
system OTF left  unfilled by sample fluorescence distribution anisotropies 
are filled, leading to  improved sampling of the system OTF particularly 
at high spatial frequencies close to $w$ and greater aberration 
sensitivity.\cite{vzurauskas2019isosense}

\begin{figure}[h]
	\centering
	\begin{subfigure}{0.4\textwidth}
		\centering
		\includegraphics[width=1\linewidth, scale=0.5]{./images/isosense_visualisation_real.png}
		\caption{}
		\label{fig:isosense_visualisation_real}
	\end{subfigure}
	\begin{subfigure}{0.4\textwidth}
		\centering
		\includegraphics[width=1\linewidth, scale=0.5]{./images/isosense_visualisation_ft.png}
		\caption{}
		\label{fig:isosense_visualisation_ft}
	\end{subfigure}
	\caption{(a) A simulated IsoSense pattern created with a 4-beam interference. A pattern similar to this is applied to the SLM (b) A diagram of a 4 beam interference pattern in Fourier space. The diagonal axis and the horizontal/vertical axis have $\frac{1}{2}$ and $\frac{1}{4}$ of the intensity of the central beam respectively.}
	\label{fig:isosense_visualisation}
\end{figure}

\section{Biological Exemplar}
\label{sec:SIM_biology}

\begin{itemize}
	\item Explain why the particular sample needs the correction provided, what you can't normally resolve without AO and show how  AO improves it.
\end{itemize}

\subsection{Experimental Setup}
\label{subsec:SIM_setup}

\subsection{Sample preparation}
\label{subsec:SIM_sample_prep}

\section{Experimental Results}
\label{sec:SIM_results}
