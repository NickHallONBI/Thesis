\chapter{Enabling Structured Illumination Microscopy at greater depth}

\section{Sample Induced Aberrations in Thick Samples}
\label{sec:sample_aberrations_thick}

\begin{itemize}
	\item Recap sample induced aberrations and how they impair SIM imaging in-particular
\end{itemize}

	\subsection{Scattering Aberrations}
	\label{subsec:scattering}
	
	\begin{itemize}
		\item Worth highlighting that at a certain point, you lose image quality to scattering aberrations which AO can't help with.
	\end{itemize}

\section{Sensorless Adaptive Optics}
\label{sec:sensorless_AO}

	\subsection{Image Quality Metric: Fourier Power}
	\label{subsec:fourier_power_metric}
	
	As mentioned in Section~\ref{subsubsec:sensorless_correction}, 
	sensorless AO correction requires an image quality metric which
	is well suited to both the imaging modality and the desired 
	sample. Microscope-AOtools implements a suite of image quality
	metrics to chose from. Since SIM data is contingent on high
	quality Fourier information, a sensible starting point for 
	the image quality metric, $S$, is the total power of the Fourier
	spectrum:
	
	\begin{equation}\label{eq:fourier_power_spectrum}
		S = \sum\limits_{n,m}{\vline \tilde{\textbf{D}}_{n,m} \vline^2}
	\end{equation}
	
	Where $\tilde{\textbf{D}}$ is the 2D discrete Fourier transform of 
	the observed fluorescent signal and $n$ and $m$ are the pixel coordinates 
	with ranges $[0, N-1]$ and $[0, M-1]$ respectively, where $N$ and $M$ are the
	number of pixels along each dimension of the image. From Parseval's Theorem, 
	$\sum\limits_{n,m}\vline \tilde{\textbf{D}}_{n,m} \vline^2$ is
	the power of the observed signal. However, the Fourier spectrum
	obtained is not entirely composed of sample information. The 
	fluorescent signal is the convolution of the distribution of fluorescent 
	emission, $E(\bar{r})$, and the system point spread function (PSF), 
	$H(\bar{r})$: 
	
	\begin{equation}\label{eq:fluor_signal_real}
		D(\bar{r}) = (E \circledast H)(\bar{r})
	\end{equation}
	
	In Fourier space, this convolution becomes $\tilde{\textbf{D}}(\bar{k}) 
	= \tilde{\textbf{E}}(\bar{k}) \tilde{\textbf{H}}(\bar{k}) = 
	\tilde{\textbf{E}}(\bar{k}) \textbf{O}(\bar{k})$, where the tilde ($\sim$)
	represent the Fourier transform of the corresponding real-space functions and 
	$\textbf{O}(\bar{k})$ is the system optical transfer function (OTF).\cite{gustafsson2008three} The OTF both attenuates the sample structure spatial frequencies and acts as
	a lowpass filter. Since $\tilde{\textbf{D}}_{n,m}$ is the 2D discrete
	Fourier tranform of the observed data, this lowpass filter, $\mu_{n,m}$, is
	defined as:
	
	\begin{equation}\label{eq:circular_mask}
	\mu_{n,m} = 
	\begin{cases}
	1, & \sqrt{n'^{2} + m'^{2}} \le w\\
	0, & \sqrt{n'^{2} + m'^{2}} > w\\ 
	\end{cases}
	\end{equation}
	
	Where $n' = n - \frac{N-1}{2}$, $m' = m - \frac{M-1}{2}$ and $w$ a radius 
	defined as the size of the field of view of the image divided by the Abbe
	diffraction limit for the system, $\frac{1.22\lambda}{2NA}$.\cite{abbe1873beitrage} 
	
	Within the OTF lowpass limit, certain spatial frequencies will be dominated
	by noise contributions. As such, when attempting to quantify the power 
	contribution from the fluorescent it is desireable to ignore the noise
	contributions. A thresholding mask, $\tau_{n,m}$ is created and defined as:
	
	\begin{equation}\label{eq:noise_threshold_mask}
	\tau_{n,m} = 
	\begin{cases}
	0, & \vline \tilde{\textbf{D}}_{n,m} \vline^2 < \delta\\
	1, & \vline \tilde{\textbf{D}}_{n,m} \vline^2 \ge \delta\\ 
	\end{cases}
	\end{equation}
	
	Where $\delta$ is a noise threshold defined as:
		
	\begin{equation}\label{eq:noise_threshold}
	\delta = \frac{\alpha}{P_{noise}}\sum\limits_{l,k}{\vline \tilde{\textbf{D}}_{l,k} \vline^2}
	\end{equation}
	
	Where $\vline \tilde{\textbf{D}}_{l,k} \vline^2$ is the power in the 
	$l$-th, $k$-th spatial frequency where $l$ \& $k$ are the coordinates 
	where $\mu_{n,m} = 0$, $P_{noise}$ is the total number of pixels 
	outside the OTF lowpass limit and $\alpha$ is a user defined 
	amplification factor. In essence, the threashold is the mean of the
	power of the spatial frequencies consisting entirely of noise signals,
	scaled by $\alpha$.	The value of $\alpha$ must be chosen to threshold
	the spatial frequencies dominated by noise without losing the sensitivity
	to true, albeit low amplitude, spatial frequencies. Emperically, an 
	$\alpha = 1.125$ offers a reasonable compromise between these traits.
	
	Finally, the bulk of the improvement in spatial frequency content will
	occur in the mid to high frequencies (relative to $w$). As such there is 
	a desire for increased sensitivity to the power fluxuations in these 
	spatial frequencies. A modulation mask, $\omega$, is constructied and 
	defined as:
	
	\begin{equation}\label{eq:attenuation_mask_norm_and_scaled}
	\omega_{n,m} = \beta \vline (1 - e^{\frac{\sqrt{n'^{2} + m'^{2}}}{w}-1}) (n'^{2} + m'^{2}) \vline
	\end{equation}
	
	Where $w$, $n'$ and $m'$ are defined as before, $\vline\textbf{x} \vline$ 
	is $\textbf{x}$ normalised to $[0,1]$, and $\beta$ is a user defined 
	normalisation factor. In effect, this modulation mask amplifies the mid 
	to high spatial frequencies while attenuating the low spatial frequencies 
	and the high spatial frequency noise close to $w$.
	
	Applying these modifaction to Equation~\ref{eq:fourier_power_spectrum} 
	yields the image quality metric:
	
	\begin{equation}\label{eq:Fourier_power_metric}
	S = \sum\limits_{n,m}{\mu_{n,m}\tau_{n,m}\omega_{n,m}\vline \tilde{\textbf{D}}_{n,m} \vline^2}
	\end{equation}
	
	Where $\mu_{n,m}$, $\tau_{n,m}$ and $\omega_{n,m}$ are defined as in
	Equations~\ref{eq:circular_mask},~\ref{eq:noise_threshold_mask} and
	\ref{eq:attenuation_mask_norm_and_scaled} respectively and 
	$\tilde{\textbf{D}}_{n,m}$ is defined as before. In practice, $S$
	measures the total power of the sample spatial frequencies, with 
	increased sensitivity to mid to high spatial frequency changes. As
	such, this image quality metric is called the Fourier Power metric.
	Since SIM imaging depends on high quality spatial frequency 
	information, this image quality metric is particularly well suited 
	to correcting aberrations for SIM data.
	
	\subsection{IsoSense}
	\label{subsec:isosense}
	
	Anisotropies in the sample structure can bias the corrections
	towards improving the image quality in a non-uniform manner.
	There has recently been a technique developed to overcome 
	this issue; IsoSense\cite{vzurauskas2019isosense}. It relies 
	on producing spatially structured light in order to fill 
	empty sections of the image Fourier spectrum. IsoSense is 
	designed to be used in structured illumination microscopy 
	(SIM) setups since they often incorporate spatial light 
	modulators (SLM) as high-speed, dynamic diffraction 
	gratings and SIM is particularly sensitive to Fourier
	space anisotropies.
	
	Microscope-AOtools incorporates the methods necessary to implement
	IsoSense. Figure~\ref{fig:isosense_visualisation} shows both
	the structured illumination pattern applied to the SLM 
	and the location of the beams in Fourier space. The illumination
	pattern shown in Figure~\ref{fig:isosense_visualisation_real} is
	the inverse Fourier transform of the 4-beam interference pattern 
	in Figure~\ref{fig:isosense_visualisation_ft}. The location 
	of these beams, $\bar{\kappa}$, are: 
	$(0,0)$, $(0,\gamma w)$, $(0,-\gamma w)$, $(\gamma w, 0)$, 
	$(-\gamma w, 0)$, $(\frac{\gamma w}{2}, \frac{\gamma w}{2})$, 
	$(-\frac{\gamma w}{2}, \frac{\gamma w}{2})$, $(\frac{\gamma w}{2},
	 -\frac{\gamma w}{2})$, $(-\frac{\gamma w}{2}, 
	 -\frac{\gamma w}{2})$. $u,v$ are the spatial frequency 
	analogues of the $x,y$ axes. $w$ is the Abbe diffraction limit and 
	$\gamma$ is a user defined fill fraction. This fill
	fraction controls the positions of the beams in the interference
	pattern and hence the region of the Fourier spectrum which will 
	be enhanced over normal illumination. The resultant fluorescent image 
	obtained, $F(x,y)$, can be described as:
	
	\begin{equation}\label{eq:isosense_real}
		F(\bar{r}) = D(\bar{r}) \times I(\bar{r})
	\end{equation}	
	
	Where $I(\bar{r})$ is the interference pattern, similar to that shown in 
	Figure\ref{fig:isosense_visualisation_real} and $E(\bar{r})$ is the sample
	structure as before. Applying a Fourier transform yeilds:
	
	\begin{equation}\label{eq:isosense_ft}
		\begin{split}
			\mathcal{F}[F(\bar{r})] &= \mathcal{F}[D(\bar{r})\times I(\bar{r})] \\
			\tilde{\textbf{F}}(\bar{k}) &= \tilde{\textbf{D}}(\bar{k}) \circledast \tilde{\textbf{I}}(\bar{k}) \\
			\tilde{\textbf{F}}(\bar{k}) &= \sum_{\bar{\kappa}}\iint\tilde{\textbf{D}}(\bar{k})\delta(\bar{k} - \bar{\kappa})dudv \\
			\tilde{\textbf{F}}(\bar{k}) &= \sum_{\bar{\kappa}}\tilde{\textbf{D}}(\bar{k} - \bar{\kappa})
		\end{split}
	\end{equation}
	
	Where $\tilde{\textbf{F}}(\bar{k})$, $\tilde{\textbf{I}}(\bar{k})$ and 
	$\tilde{\textbf{D}}(\bar{k})$ are the Fourier transforms of $F(\bar{r})$, 
	$I(\bar{r})$ and $D(\bar{r})$, the $\bar{\kappa}$ frequencies are the 
	locations of interference pattern beams described above. By convolving 
	the sample structure with multiple $\delta$-functions, multiple copies 
	of the sample spatial frequency information are created and regions of 
	the system OTF left unfilled by sample structure anisotropies are 
	filled, leading to improved sampling of the system OTF particularly
	at high spatial frequencies close to $w$ and greater aberration
	sensitivity.\cite{vzurauskas2019isosense}
	
	\begin{figure}[h]
		\centering
		\begin{subfigure}{0.4\textwidth}
			\centering
			\includegraphics[width=1\linewidth, scale=0.5]{./images/isosense_visualisation_real.png}
			\caption{}
			\label{fig:isosense_visualisation_real}
		\end{subfigure}
		\begin{subfigure}{0.4\textwidth}
			\centering
			\includegraphics[width=1\linewidth, scale=0.5]{./images/isosense_visualisation_ft.png}
			\caption{}
			\label{fig:isosense_visualisation_ft}
		\end{subfigure}
		\caption{(a) A simulated IsoSense pattern created with a 4-beam interference. A pattern similar to this is applied to the SLM (b) A diagram of a 4 beam interference pattern in Fourier space. The diagonal axis and the horizontal/vertical axis have $\frac{1}{2}$ and $\frac{1}{4}$ of the intensity of the central beam respectively.}
		\label{fig:isosense_visualisation}
	\end{figure}

\section{Biological Exemplar}
\label{sec:SIM_biology}

\begin{itemize}
	\item Explain why the particular sample needs the correction provided, what you can't normally resolve without AO and show how  AO improves it.
\end{itemize}

	\subsection{Experimental Setup}
	\label{subsec:SIM_setup}

	\subsection{Sample preperation}
	\label{subsec:SIM_sample_prep}
	
\section{Experimental Results}
\label{sec:SIM_results}