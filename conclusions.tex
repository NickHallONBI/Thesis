\chapter{Conclusions}
\label{chpt:conclusions}

\section{Summary of Contributions}
\label{sec:contributions}

The primary contributions of this thesis surround enabling the successful 
deployment of AO in microscopy, culminating in enabling AO in 
super-resolution systems for improved image quality and imaging depth. As 
outlined in Figure~\ref{fig:ao_system_setup_workflow} this process consists 
of four phases:

\begin{enumerate}
	\item \textit{System Design Phase}: A potential user should consider 
	the needs of their imaging modality, system constraints, desired 
	sample types before deciding on the appropriate AO element to 
	implement.
	\item \textit{Installation Phase}: The user installs the chosen AO 
	element into their beam path.
	\item \textit{Set-up Phase}: The AO element is calibrated to correct 
	for optical aberrations. This calibration is checked and the system 
	aberrations are corrected.
	\item \textit{Sample Correction Phase}: The sample correction routine 
	is designed. This will typically fall into one of two categories; 
	sensorless AO or direct wavefront sensing.
\end{enumerate}  

The development of the alignment tool BeamDelta improves the accuracy of the 
\textit{Installation Phase} by providing an empirical accurate measure of 
relative alignment between beam paths\cite{dobbie2019beamdelta}. It also aids 
in aberration correction indirectly by minimising the initial optical 
aberrations present due to alignment errors.

The development of Microscope-AOtools provides a robust, generalised 
software implementation for the methods required by the \textit{Set-up Phase} 
and \textit{Sample Correction Phase}\cite{hall2020microscope}. 
Microscope-AOtools is designed to be modular and configurable in order to 
enable it to be used on a range of microscopes with varying adaptive 
elements, wavefront sensing techniques and sample correction techniques. 
Microscope-AOtools is integrating into Python-Microscope in order to enable 
accessibility to microscope users without requiring a complete understanding 
of all the technical details implemented.

The primary evidence for the claim that Microscope-AOtools is robust and 
generalised is its successful deployment on two separate imaging systems, 
with additional deployments forthcoming. On both the Aurox Clarity and 
DeepSIM systems, AO correction provided by Microscope-AOtools yielded 
significant increases in image quality and contrast. On the DeepSIM system, 
the AO correction was able to enable diffraction-limited imaging of a highly 
regular fluorescent structure deep in a glass structure. The AO correction 
also enabled 3D SIM imaging up to 30$\mu$m in biological samples, well beyond 
the typical 10-20$\mu$m imaging depth for 
SIM\cite{schermelleh2019super,wu2018faster}. 

\section{Final Conclusion}
\label{sec:final_conclusion}

For some time, there has been a call for a robust, generalised implementation 
for AO. Such an implementation should incorporate all the methods needed to
setup and operate an AO element for a range of imaging modalities and sample 
types. These methods should be exposed to the users in an accessible fashion 
to allow easy-of-use and require minimal direct user involvement. 
Microscope-AOtools includes methods for calibration, direct wavefront and 
sensorless correction. In particular, it already incorporates several image 
quality metrics suited to sensorless correction in a number of different 
imaging modalities. It also includes a characterisation method for assessing 
the accuracy of the calibration step. It has also been designed in a modular 
manner allowing for new wavefront sensing techniques and image quality 
metrics to be added with minimal disruption to the rest of the workflows and, 
therefore, minimal work duplication. Microscope-AOtools has been integrated 
into the Microscope-Cockpit user interface in an accessible manner which 
allows for a user to control key parameters whilst obfuscating much of the 
technical complexity. It has been shown to be successfully deployed on a 
number of different systems, with additional deployments forthcoming. With 
time and community support, such an implementation has scope to go beyond its 
current state of ``generalised software implementation'' and become a universal software implementation for AO.
