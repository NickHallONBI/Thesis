\chapter{Conclusions}
\label{chpt:conclusions}

\section{Summary of Contributions}
\label{sec:contributions}

The primary contributions of this thesis surround enabling the successful 
deployment of AO in microscopy, culminating in enabling AO in 
super-resolution systems for improved image quality and imaging depth. As 
outlined in Figure~\ref{fig:ao_system_setup_workflow} this process consists 
of four phases:

\begin{enumerate}
	\item \textit{System Design Phase}: A potential user should consider 
	the needs of their imaging modality, system constraints, desired 
	sample types before deciding on the appropriate AO element to 
	implement.
	\item \textit{Installation Phase}: The user installs the chosen AO 
	element into their beam path.
	\item \textit{Set-up Phase}: The AO element is calibrated to correct 
	for optical aberrations. This calibration is checked and the system 
	aberrations are corrected.
	\item \textit{Sample Correction Phase}: The sample correction routine 
	is designed. This will typically fall into one of two categories: 
	sensorless AO or direct wavefront sensing.
\end{enumerate}  

The development of the alignment tool BeamDelta improves the accuracy of the 
\textit{Installation Phase} by providing an empirical accurate measure of 
relative alignment between beam paths\cite{dobbie2019beamdelta}. It also aids 
in aberration correction indirectly by minimising the initial optical 
aberrations present due to alignment errors.

The development of Microscope-AOtools provides a robust, generalised 
software implementation for the methods required by the \textit{Set-up Phase} 
and \textit{Sample Correction Phase}\cite{hall2020microscope}. 
Microscope-AOtools is designed to be modular and configurable in order to 
enable it to be used on a range of microscopes with varying adaptive 
elements, wavefront sensing techniques and sample correction techniques. 
Microscope-AOtools is integrating into Python-Microscope in order to enable 
accessibility to microscope users without requiring a complete understanding 
of all the technical details of the implementation and wide
applicability due to the varied hardware support.

The primary evidence for the claim that Microscope-AOtools is robust and 
generalised is its successful deployment on two separate imaging systems, 
with additional deployments forthcoming. On both the Aurox Clarity and 
DeepSIM systems, AO correction provided by Microscope-AOtools yielded 
significant increases in image quality and contrast. On the DeepSIM system, 
the AO correction was able to enable diffraction-limited imaging of a highly 
regular fluorescent structure deep in a glass structure with
significant spherical aberration. The AO correction 
also enabled 3D SIM imaging up to 30$\mu$m in biological samples, well beyond 
the typical 10-20$\mu$m imaging depth for 
SIM\cite{schermelleh2019super,wu2018faster}. 

\section{Future development}
\label{sec:future_dev}

Chapter~\ref{chpt:ao_tools} discussed the specific methods implemented in 
Microscope-AOtools for both AO device set up and operation. These Set-up 
and  Sample Correction methods rely on suites of techniques, wavefront 
sensing and image quality metric assessment respectively. These are 
designed to be easily extensible by users as new techniques are developed. 
The functions defining the existing wavefront sensing and image quality 
metric assessment techniques are stored in separate files. Which wavefront 
sensing technique will be used is an attribute of the highest level
of the code hierarchy and is used to select a wavefront sensing technique 
from the unwrapping method dictionary. Similarly, which image quality 
metric assessment will be used is an attribute of a lower level	of the 
code hierarchy and is used to select a wavefront sensing technique from 
the dictionary of image quality metrics. A detailed guide of where the 
appropriate classes and dictionaries are located and how to add new 
wavefront sensing and image quality metric assessment techniques is 
included in the \textit{README.md} file for Microscope-AOtools. Briefly, 
these suites are composed of functions with a defined set of input and 
output variables. A user creates a new wavefront sensing or image 
quality metric assessment functions with the input and outputs defined 
in the \textit{README.md}, adds this function to the correct file and 
then adds the function option to the correct suite dictionary.

Microscope-AOtools has been designed so a user can take an adaptive element 
in an arbitrary set-up, calibrate the adaptive element and use it on any 
sample type in a range of imaging modalities. Since Microscope-AOtools 
leverages Python-Microscope it already supports a number of adaptive 
elements, mostly deformable mirrors, which will expand as hardware support 
in Python-Microscope expands. Adding new devices to Python-Microscope is 
relatively simple. Refer to Python-Microscope 
(\url{https://www.python-microscope.org/}) for more details. 
Microscope-AOtools only requires that the adaptive element be a 
Python-Microscope device which has an attribute \textit{n\textunderscore 
	actuators} which defines the number of variable components of the device.

The process of setting up an adaptive element requires a wavefront sensor 
to observe the shape of the phase wavefront and calibrate how the variable 
components of the adaptive element affect this wavefront. By designing the 
set-up methods in Microscope-AOtools to accept any method from a suite of 
wavefront sensing techniques, Microscope-AOtools is both generalised and 
easily extensible. If the desired wavefront sensing technique is not 
already incorporated then a user only has to add the function necessary to 
perform the wavefront sensing step rather than to reimplement the set-up 
methods in their entirety. Swapping between wavefront sensing techniques is 
not implemented in the Cockpit user interface (UI) for AO control since the 
sensing technique is generally reliant on the physical setup and is 
unlikely to change between experiments.

Microscope-AOtools is further generalised as it allows for the control 
matrix to be acquired by some external method and then set in 
Microscope-AOtools with the \textit{set\textunderscore controlMatrix} 
method. This ensures that a user with an existing calibration routine 
wishing to access the sensorless AO methods in Microscope-AOtools can do so 
without having to repeat work they have already performed. It also allows 
control matrices acquired from routines using different phase acquisition 
techniques to be compared. Characterisation assays can be acquired for each 
method's control matrix and the accuracy of the Zernike mode recreation 
compared.

The generalised nature of Microscope-AOtools continues into the Sample 
Correction methods. By allowing the user to swap between wavefront sensing 
techniques, Microscope-AOtools already possesses all the methods necessary 
for performing sample correction using direct wavefront sensing, provided 
the wavefront sensing technique is already included in the suite of 
methods. Similarly, Microscope-AOtools utilises a suite of image quality 
metrics suited to different sample types and imaging modalities. A user can 
select a pre-existing metric well suited to their application. If no 
appropriate metric currently exists a new one can easily be implemented 
and added  to the suite of metrics. Once implemented it can be used in any 
of the sensorless AO analysis methods outlined in 
Figure~\ref{fig:sensorless_correction_method}. Since the best image quality 
metric can vary between samples, this variable is left configurable from 
the Cockpit UI as shown in Figure~\ref{fig:DM_methods_cockpit_options}. A 
user need only add the appropriate label for the newly implemented image 
quality metric assessment technique to the options present in the Cockpit 
device code and it will be available with the other techniques shown in 
Figure~\ref{fig:DM_methods_cockpit_options}.

Furthermore, Microscope-AOtools allows for Zernike mode amplitudes to be 
set directly with the \textit{set\textunderscore phase} method. This means 
that if a user has an offline analysis technique, such as a machine 
learning approach, Microscope-AOtools can be used to calibrate the 
deformable mirror, the sample induced calculations are performed offline 
and then the appropriate correction applied through Microscope-AOtools.

Microscope-AOtools is free and open-source. It is intended to be a resource 
for the microscopy community at large and it is designed to minimise the 
time and effort spent replicating work other AO users have already done. As 
Microscope-AOtools acquires a larger base of users, some adding their own 
wavefront sensing techniques or image quality metrics to expand the 
existing suites, future and existing users will have a wider array of 
usability options, accelerating the adoption of novel techniques by the 
microscopy community and lowering the barrier to entry to set-up an AO 
system.

Beyond the open-ended task of expanding the existing suite of phase 
acquisition techniques and image quality metrics, there are a number of 
future developments that could be made to Microscope-AOtools. There does 
not currently exist a universal image quality metric, although strides have 
been made in that direction\cite{antonello2020multi}. Image quality metrics 
attempt to assign a numerical value for how `good' an image is, but what 
makes a `good' image varies between imaging modalities, sample type and 
even users. Most metrics pick some aspect of the image deemed to be 
significant (e.g. contrast, sharpness, maximum intensity, etc) and maximise 
it. Since Microscope-AOtools has access to multiple image quality metrics, 
one development would be designing a sensorless AO routine which measures 
multiple image qualities simultaneously, assigns some weight to each metric 
measurement and maximises the image quality based on several criteria.

Another direction for further development concerns restructuring the 
sensorless correction workflow. As explained in 
Section~\ref{subsec:sensorless_correction}, each Zernike mode is corrected 
for separately. There is an underlying assumption here that since the 
Zernike modes themselves are an orthonormal basis set, the effect they have 
on the image quality must also be orthogonal. However, phenomena such as 
the focal shift which accompanies spherical aberration would suggest that 
this is not necessarily the case\cite{torok1997role}. It may therefore be 
possible to obtain the optimum image quality faster and with fewer images 
utilising the family of solvers used for objective function maximisation, 
such as the Nelder-Mead method, than the current uni-variant 
approach\cite{nelder1965simplex}. Such an approach has been shown to work 
in principle on point objects to recover diffraction limited images but has 
not yet been demonstrated on biological samples\cite{murray2005wavefront}.

Finally, the integration of adaptive optical elements, primarily deformable 
mirrors, into Cockpit allows for the construction of remote focusing 
experiments. In remote focusing experiments, the adaptive element is used 
to shift the focal plane by applying a known amount of optical aberrations 
to the beam path, instead of manually moving the sample plane. This allows 
for higher axial scan speeds - since deformable mirrors are, on average, an 
order of magnitude faster than mechanical stages - and avoids mechanical 
interference between the objective lens and the sample through dipping 
media such as water or oil\cite{botcherby2008optical,vzurauskas2017rapid} 
This would require constructing another calibration routine to obtain the 
maps between physical displacement and defocus aberration applied in order 
to ensure accurate and reliable remote focusing.

\section{Final Conclusion}
\label{sec:final_conclusion}

For some time, there has been a call for a robust, generalised implementation 
for AO. Such an implementation should incorporate all the methods needed to
setup and operate an AO element for a range of imaging modalities and sample 
types. These methods should be exposed to the users in an accessible fashion 
to allow easy-of-use and require minimal direct user involvement. 
Microscope-AOtools includes methods for calibration, direct wavefront and 
sensorless correction. In particular, it already incorporates several image 
quality metrics suited to sensorless correction in a number of different 
imaging modalities. It also includes a characterisation method for assessing 
the accuracy of the calibration step. It has also been designed in a modular 
manner allowing for new wavefront sensing techniques and image quality 
metrics to be added with minimal disruption to the rest of the workflows and, 
therefore, minimal work duplication. Microscope-AOtools has been integrated 
into the Microscope-Cockpit user interface in an accessible manner which 
allows for a user to control key parameters whilst obfuscating much of the 
technical complexity. It has been shown to be successfully deployed on a 
number of different systems, with additional deployments forthcoming. With 
time and community support, such an implementation has scope to go beyond its 
current state of ``generalised software implementation'' and become a
universal software implementation for AO.

(XX should say you have demonstarted that it has significantly
improved imaging iusing it, especially in challenging conditions, deep
in biolofgical tissue. XX).
