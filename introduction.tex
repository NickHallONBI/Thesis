\chapter{Introduction}

\section{Introduction to Microscopy}
\label{sec:microscopy}

\begin{itemize}
	\item This is the section the give a brief overview of geometric optics, microscopy, fluorescence and, particularly, resolution.
\end{itemize}

\section{Super-resolution Microscopy}
\label{sec:super_res}

\begin{itemize}
	\item Introduction to the fact that there are methods for overcoming the resolution limit and different techniques with different pros/cons
\end{itemize}

	\subsection{Structured Illumination Microscopy}
	\label{subsec:SIM}
	
		\begin{itemize}
			\item General overview of SIM, particularly Gustafsson-style SIM.
		\end{itemize}
	
	%\subsection{SIMFlux Microscopy}
	%\label{subsec:SIMFlux}
	
	%	\begin{itemize}
	%		\item Overview of SIMFlux
	%	\end{itemize}

\section{Optical Aberrations}
\label{sec:aberrations}

\begin{itemize}
	\item Talk about the effect optical aberration have on PSF quality and therefore image quality/resolution. Also worth talking about optical aberrations and ``non-optical'' aberrations i.e. piston, tip and tilt.
\end{itemize}

\section{Adaptive Optics}
\label{sec:AO}

\begin{itemize}
	\item Here we introduce adaptive optics and their already proven usefulness in correcting for aberrations in microscopy.
	\item Raise but don't address the non-generalised nature of current implementations
\end{itemize}

\section{Overview and Aims of the Thesis}
\label{sec:overview}

\begin{itemize}
	\item Here is where you should really hammer home that current implementations are a) not generalised and b) not accessible to biogists.
	\item Present the story of the thesis: "I have created a robust, generalised accessible adaptive optics implementation. It solves a range of common biology problems and works across multiple sample types and imaging modality. It's generalised so it can be transferred between systems." Thats the novel contribution.
\end{itemize}
