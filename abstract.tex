\begin{abstract}
	
	Many of the recent innovations in biological imaging have revolved around the 
	quest for greater resolving power, ultimately culminating in the advent of
	super-resolution microscopy techniques. However, all microscopy techniques are
	vulnerable to optical aberrations which distort the light wavefront. This leads 
	to a gulf between theoretical and practical resolution for an imaging system. 
	For super-resolution techniques, this can lead to reconstruction artifacts or 
	the failure of the imaging technique entirely.
	
	Implementing adaptive optics (AO) in microscopy has already been shown to be 
	highly effective at reducing  these aberrations and yielding significant 
	improvements to image quality and resolution in numerous proof of principle 
	systems. Despite this, AO technology has yet to be widely adopted in 
	microscopy. This is for two principle reasons. Firstly, AO implementations to 
	date have not been robust or generalised which makes transferring them between 
	microscopy systems, imaging modalities and sample type troublesome to 
	impossible. Secondly, AO implementations to date have not been accessible to 
	typical microscope users and instead have been the purview of AO microscopy 
	specialists.
	
	This thesis presents a generalised, robust implementation for AO; 
	Microscope-AOtools. This implementation has all the necessary methods for 
	setting up and operating an adaptive element in microscopy. It has a flexible, 
	modular design which allows for easy transfer between imaging systems, 
	modalities, hardware configurations and sample types. These methods are 
	integrated into Microscope-Cockpit for user accessibility. The evidence of 
	these claims are substantiated by a detailed description of Microscope-AOtools' 
	successful deployment on both a spinning disk confocal system and a bespoke, 
	upright structured illumination microscope with a range of sample types.
	
\end{abstract}
