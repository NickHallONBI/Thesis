\chapter{Point object seperation figure generation}

Figures~\ref{fig:Airy_ring_2_object_seperation} and 
\ref{fig:aberrations_res} were generated using the code 
presented in Sections~\ref{sec:ideal_po_sep_code} and 
\ref{sec:aberrated_po_sep_code} respectively. The majority of 
imported packages are freely available through PyPI. 
czernike is part of the enzpy Python package. The fourier 
package is used to generate aberrated PSFs using regular
Fourier optics\cite{schmidt2010numerical} and real-valued 
Zernike coefficients in the pupil only. It was developed and 
provided by Dr Jacopo Antonello.

\section{Ideal point object seperation}
\label{sec:ideal_po_sep_code}

\begin{lstlisting}[language=Python]
#!/usr/bin/env python3
# -*- coding: utf-8 -*-
	
import sys  # noqa
import numpy as np  # noqa
import matplotlib.pyplot as plt  # noqa
	
from h5py import File
	
from czernike import RZern  # noqa
from fourier import Fraunhofer  # noqa
	
plt.close('all')
	
Np = 256
wavelength = 561.0e-9
NA = 1.1
n1 = 1.51
d = 1.22*wavelength/(2*NA)
pixel_size = d/32
crit_fracs = [0.5,1.0,1.22,1.5,2.0]
n_alpha = 8  # maximum radial order of Zernike polynomials
	
v = np.linspace(-((pixel_size*(Np/2))/d)*np.pi,
((pixel_size*(Np/2))/d)*np.pi, Np)
	
# load an object
#with File('./obj.h5', 'r') as f:
#    obj = f['obj'][()]
it = 0
	
for crit_frac in crit_fracs:
	
	point_obj_1 = np.zeros((Np,Np))
	point_obj_1[Np//2, Np//2] = 1
	
	point_obj_2 = np.zeros((Np,Np))
	point_obj_2[Np//2, Np//2 + 
		int((crit_frac*wavelength)/(2*NA*pixel_size))] = 1
	
	obj = np.zeros((Np,Np))
	obj[Np//2, Np//2] = 1
	obj[Np//2, Np//2 + 
		int((crit_frac*wavelength)/(2*NA*pixel_size))] = 1
	
	# object to make PSFs using Fourier optics
	fh = Fraunhofer(wavelength=wavelength, NA=NA, N=Np,
		pixel_size=pixel_size, n_alpha=n_alpha)
	
	# random aberration
	alpha = fh.make_random_aberration(rms=1)
	
	# uncomment for fixed aberration
	alpha *= 0
	noll = 1
	alpha[noll - 1] = 0
	
	psf = fh.psf(alpha)
	mtf = fh.mtf(psf)
	
	point_obj_1_img = fh.incoherent(point_obj_1, psf)
	point_obj_1_img_norm = 
		point_obj_1_img/np.max(point_obj_1_img)
	
	point_obj_2_img = fh.incoherent(point_obj_2, psf)
	point_obj_2_img_norm = 
		point_obj_2_img/np.max(point_obj_2_img)
	
	img = fh.incoherent(obj, psf)
	img_norm = img/np.max(img)
	
	display_min = np.min(img_norm)
	display_max = np.max(img_norm)
	
	plt.figure(it*2,figsize=(10,3.85))
	nn = 1
	mm = 2
	
	plt.subplot(nn, mm, 1)
	plt.plot(v, point_obj_1_img_norm[Np//2,:], 'r--')
	plt.plot(v, point_obj_2_img_norm[Np//2,:], 'b--')
	plt.plot(v, (point_obj_1_img_norm+
		point_obj_2_img_norm)[Np//2,:], 'g')
	plt.ylabel("Normalised intensity")
	plt.xlabel("Normalised lateral coordinate")
	#plt.title("Object separation %.3f\u03C0" %crit_frac)
	#plt.tight_layout()
	
	plt.subplot(nn, mm, 2)
	plt.imshow(img)
	plt.tick_params(axis='both', which='both', 
		bottom=False, top=False, labelbottom=False, 
		right=False, left=False, labelleft=False)
	
	plt.tight_layout()
	plt.savefig("Airy_ring_2_object_seperation_%s_%s" 
		%(str(crit_frac)[0],str(crit_frac)[2:]), 
		bbox_inches='tight')
	
	it += 1
	
plt.show()
\end{lstlisting}

\section{Aberrated point object seperation}
\label{sec:aberrated_po_sep_code}

\begin{lstlisting}[language=Python]
#!/usr/bin/env python3
# -*- coding: utf-8 -*-

import sys  # noqa
import numpy as np  # noqa
import matplotlib.pyplot as plt  # noqa

from h5py import File

from czernike import RZern  # noqa
from fourier import Fraunhofer  # noqa

plt.close('all')

Np = 256
wavelength = 561.0e-9
NA = 1.1
n1 = 1.51
d = 1.22*wavelength/(2*NA)
pixel_size = d/32
crit_frac = 1.22
n_alpha = 8  # maximum radial order of Zernike polynomials

v = np.linspace(-3*np.pi, 3*np.pi, Np)

# load an object
#with File('./obj.h5', 'r') as f:
#    obj = f['obj'][()]

point_obj_1 = np.zeros((Np,Np))
point_obj_1[Np//2, Np//2] = 1

point_obj_2 = np.zeros((Np,Np))
point_obj_2[Np//2, Np//2 + 
	int((crit_frac*wavelength)/(2*NA*pixel_size))] = 1

obj = np.zeros((Np,Np))
obj[Np//2, Np//2] = 1
obj[Np//2, Np//2 + 
	int((crit_frac*wavelength)/(2*NA*pixel_size))] = 1

# object to make PSFs using Fourier optics
fh = Fraunhofer(
wavelength=wavelength, NA=NA, N=Np, pixel_size=pixel_size,
n_alpha=n_alpha)

# random aberration
alpha = fh.make_random_aberration(rms=1)

# uncomment for fixed aberration
alpha *= 0
noll = 1
alpha[noll - 1] = 0

psf = fh.psf(alpha)
mtf = fh.mtf(psf)

point_obj_1_img = fh.incoherent(point_obj_1, psf)
point_obj_1_img_norm = point_obj_1_img/np.max(point_obj_1_img)

point_obj_2_img = fh.incoherent(point_obj_2, psf)
point_obj_2_img_norm = point_obj_2_img/np.max(point_obj_2_img)

img = fh.incoherent(obj, psf)
img_norm = img/np.max(img)

display_min = np.min(img_norm)
display_max = np.max(img_norm)

# random aberration
alpha_ab = fh.make_random_aberration(rms=1)

it = 0
for noll in range(1,12):
#for noll in range(1):

	# uncomment for fixed aberration
	alpha_ab *= 0
	alpha_ab[noll - 1] = 1

	psf_ab = fh.psf(alpha_ab)
	mtf_ab = fh.mtf(psf_ab)

	point_obj_1_img_ab = fh.incoherent(point_obj_1, psf_ab)
	point_obj_1_img_norm_ab = 
		point_obj_1_img_ab/np.max(point_obj_1_img)

	point_obj_2_img_ab = fh.incoherent(point_obj_2, psf_ab)
	point_obj_2_img_norm_ab = 
		point_obj_2_img_ab/np.max(point_obj_2_img)

	img_ab = fh.incoherent(obj, psf_ab)
	img_norm_ab = img_ab/np.max(img)

	plt.figure(it*2)
	nn = 2
	mm = 2

	plt.subplot(nn, mm, 1)
	plt.plot(v, point_obj_1_img_norm_ab[Np//2,:])
	plt.ylabel("Normalised intensity")
	plt.xlabel("Normalised lateral coordinate")

	plt.subplot(nn, mm, 2)
	plt.imshow(point_obj_1_img_norm_ab, vmin=display_min, 
		vmax=display_max)
	plt.tick_params(axis='both', which='both', bottom=False, 
		top=False, labelbottom=False, right=False, 
		left=False, labelleft=False)
	plt.title('PSF')

	plt.subplot(nn, mm, 3)
	plt.plot(v, point_obj_1_img_norm_ab[Np//2,:], 'r--')
	plt.plot(v, point_obj_2_img_norm_ab[Np//2,:], 'b--')
	plt.plot(v, 
		(point_obj_1_img_norm_ab+point_obj_2_img_norm_ab)
		[Np//2,:], 'g')
	plt.ylabel("Normalised intensity")
	plt.xlabel("Normalised lateral coordinate")
	#plt.title("Object separation %.3f\u03C0" %crit_frac)

	plt.subplot(nn, mm, 4)
	plt.imshow(img_norm_ab, vmin=display_min, 
	vmax=display_max)
	plt.tick_params(axis='both', which='both', bottom=False, 
		top=False, labelbottom=False, right=False, 
		left=False, labelleft=False)
	plt.title('Image')

	plt.tight_layout()
	plt.savefig("Airy_ring_2D_2_object_seperation_%s_%s_Noll_%i" 
	%(str(crit_frac)[0],str(crit_frac)[2:],noll), 
	bbox_inches='tight')

	plt.figure((it*2)+1)
	nn = 2
	mm = 2

	plt.subplot(nn, mm, 1)
	plt.plot(v, point_obj_1_img_norm[Np//2,:], 'r--')
	plt.plot(v, point_obj_2_img_norm[Np//2,:], 'b--')
	plt.plot(v, 
		(point_obj_1_img_norm+point_obj_2_img_norm)[Np//2,:]
		,'g')
	plt.ylabel("Normalised intensity")
	plt.xlabel("Normalised lateral coordinate")
	#plt.title("Object separation %.3f\u03C0" %crit_frac)

	plt.subplot(nn, mm, 2)
	plt.imshow(img_norm, vmin=display_min, vmax=display_max)
	plt.tick_params(axis='both', which='both', bottom=False, 
		top=False, labelbottom=False, right=False, 
		left=False, labelleft=False)
	plt.title('Unaberrated image')

	plt.subplot(nn, mm, 3)
	plt.plot(v, point_obj_1_img_norm_ab[Np//2,:], 'r--')
	plt.plot(v, point_obj_2_img_norm_ab[Np//2,:], 'b--')
	plt.plot(v, 
		(point_obj_1_img_norm_ab+point_obj_2_img_norm_ab)
		[Np//2,:], 'g')
	plt.ylabel("Normalised intensity")
	plt.xlabel("Normalised lateral coordinate")
	#plt.title("Object separation %.3f\u03C0" %crit_frac)

	plt.subplot(nn, mm, 4)
	plt.imshow(img_norm_ab, vmin=display_min, 
		vmax=display_max)
	plt.tick_params(axis='both', which='both', bottom=False, 
		top=False, labelbottom=False, right=False, 
		left=False, labelleft=False)
	plt.title('Aberrated image')

	plt.tight_layout()
	plt.savefig("Airy_ring_2D_2_object_seperation_aberration_c
	omparison_Noll_%i" %noll, bbox_inches='tight')

	it += 1

plt.show()
\end{lstlisting}