\chapter{Aberration Correction in for Spinning Disk Confocal Microscopy}
\label{chpt:Aurox}

\section{Image Formation in Aurox Clarity Module}
\label{sec:Aurox_image_formation}

	\subsection{Aberrations and Image Formation}
	\label{subsec:Aurox_aberrations}

\section{Biological Exemplar}
\label{sec:Aurox_biology}

	\begin{itemize}
		\item Explain why the particular sample needs the correction provided, what you can't normally resolve without AO and show how  AO improves it.
	\end{itemize}

	\subsection{Experimental Setup}
	\label{subsec:Aurox_setup}

	\subsection{Sample preperation}
	\label{subsec:Aurox_sample_prep}
	
\section{Experimental Results}
\label{sec:Aurox_results}
